%%
% 15_creating-the-earth.tex
%
% @author: Patrick Scherler
%%
\newpage
\section{Creating the Earth}

\subsection*{1. What is the leading question of this chapter? (p. 185)}
How could E=mc2 operate to create the ordinary elements of our planet and daily life as well?

\subsection*{2. How did Hoyle explain the creation of the elements?}
If a star ever imploded, its center could reach close to 100 million degrees. That would be enough to squeeze even the larger nuclei of more massive elements together.

\subsection*{3. Where did he get his inspiration for the theory of implosion?}
He found out that the scientists of the Manhattan Project use implosion to get a full nuclear reaction of the plutonium when the bomb explodes. Implosion raises the pressure and temperature enough to do that.

\subsection*{4. What keeps our planet hot at the core and causes continental shifts (earthquakes)?}

This high temperatures underneath the surface shift the thin continents on top to shape the surface of Earth.

\subsection*{5. What examples of modern human applications of E=mc2 are given?}
Atomic bombs were one of the first direct applications. Then nuclear submarines and power stations were built. E=mc2 continues at work in ordinary houses (smoke detectors, exit signs), hospitals (PET scans) and the archaeology (C-14 clock).