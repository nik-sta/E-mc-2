%%
% 12_americas-turn.tex
%
% @author: Patrick Scherler
%%
\newpage
\section{America's Turn}

\subsection*{1. Who was appointed as the day-to-day manager of the Los Alamos bomb construction team?}
The exquisitely oversensitive \emph{J. Robert Oppenheimer} was selected to be in \emph{day-to-day control} of the scientists at Los Alamos.

The man who was appointed to the \emph{overall charge} of the atomic bomb program, was \emph{Leslie Groves}. He was effective in getting things built. Groves got all the required factories and the vast reactor done on time and under budget. But he would almost certainly have failed at inspiring theoreticians in unexplored intellectual terrain. Luckily he had chosen Robert Oppenheimer for this task.

\subsection*{2. What was Oppenheimer like and how did he motivate his team?}
Still in his twenties he become one of America's top theoretical physicists. He seemed effortlessly good at everything. Oppenheimer was superb at identifying weaknesses or inner doubts in others. This ability to detect other people's deepest fears made him a perfect leader.

\subsection*{3. Which two ways to build a bomb did America pursue?}
One team took a blunt approach and was simply trying to extract the most explosive component in natural uranium. When enough of that was accumulated, there'd be a bomb.

Another team was taking a more subtle approach. They were starting with ordinary uranium, and then hoping to transform it into the wickedly powerful, new plutonium metal.

\subsection*{4. How did the scientists try to make Plutonium explode?}
The idea was to start with a ball of plutonium that was fairly low density. That wouldn't explode. But then you'd wrap explosives around it, and set them off, all at precisely the same instant.

\subsection*{5. What news did Niels Bohr bring from Europe in February 1944?}
The Germans were very close to building the bomb. They repaired the Vemork factory in Norway to produce heavy water for their atomic bomb.

\subsection*{6. What kind of sabotage did the Norwegian Haukelid do?}
He sank the ferry which should bring the heavy water from Norway to Germany.

\subsection*{7. What did finally happen to Germany's bomb building capacity?}
By the end, the German researchers had reached about half the rate of nucleus splitting needed for a sustained chain reaction. Heisenberg knew he wouldn't get further.

\subsection*{8. What happened to Heisenberg?}
He became a prisoner of war, and returned to being professor of physics in Germany in 1946.

Heisenberg would be welcomed as a hero in Germany when he was finally released in 1946, while Oppenheimer, even before the war ended, knew his postwar life wouldn't be so simple.

\subsection*{9. What happened to Oppenheimer after the war?}
He was completely monitored by the FBI and he regretted the use of the bomb.


\subsection*{10. Which two opposing views were there in the US concerning the deployment of the bomb in Japan?}
The initial plan was to use the bomb on city of Japan and force them to surrender.

The feeling it might not be needed was so strong that there was talk about having demonstrations first, or at least adjusting the phrasing in the surrender demands to make clear that the emperor could remain in place.

\subsection*{11. Why was the bomb deployed?}
President Truman's most forceful adviser \emph{Jimmy Byrnes} convinced him to use the bomb.
