%%
% m-is-for-mass.tex
%
% @author: Nikola Stankovic
%%

\section{m is for mass}

\subsection*{How is Lavoisier’s character described?}
He has keen sense of justice. He rescued an innocent thirteen-year-old daughter, Marie Anne, of his friend and boss Jacques Paulze from a forced marriage to an uncouth, gloomy, rich ogre of man by marrying her himself.\emph{Lavoisier} was hard working, six days a week and had a obsession with careful measuring. So he was a romantic person with a great sense of finicky precision.


\subsection*{What did Lavoisier discover when analyzing the burning/rusting of metals?}
He wanted to find out whether the small piece of metal, which slowly burn or rust, would weigh more or less that it did before. To find the answer he built a entirely closed apparatus. With a huge amount of repetitions he has found out that  a rusted sample does not weigh less. It doesen't weigh the same. \emph{It weighs more.} He discovered that some of the gases must habe flown down and stuck to the metal. That was the extra weight he had found.

\subsection*{Which famous scientist had already developed a theory of mass and movement in the 1600s?}
\emph{Isaac Newton}. He has shown that all the planets, moons and comets we see could be described as being cranked along inside an immense God-created machine. Lavoisier was the right man to find out if Newton's vision really did apply on Earth.

\subsection*{Why was Lavoisier so unpopular during the French revolution?}
Lavoisier was so unpopular during the French revolution, because he decided to build a wall around Paris, a missive one, where everyone could be stopped, searched, and forced to pay tax. He made this decision a short time before the start of the french revolution. The Parisians hated it, and when the Revolution began, it was the first large structure they attacked. Additional he had with the Swiss-born Dr. Jean-Paul Marat a hidden enemy. Hidden because Lavoisier forgot that he have turned him down, but \emph{Marat} never forgot. By 1793, Jean-Paul Marat was head of a leading faction in the National Assembly. He'd suffered years of poverty because of Lavoisier's rejection. Marat didn't kill him immediately. Instead, he made sire Paris's citizens were constantly reminded of the wall. In November 1793 Lavoisier was arrested in the Louvre and executed with Dr. Guillotin's instrument as the fourth victim, just after his friend Paulze.

\subsection*{What was Einstein taught in the 1890s about mass and energy?}
By the mid-1800s, scientists accepted the vision of energy and mass as being like two separate domed cities. Everyone thought that nothing connected the two realms. That was what Einstein was thought in the 1890s: that energy and mass were different topics, that they had nothing to do with each other.

\subsection*{How did Einstein establish a link between the two domains?}
Einstein did find that that there was a link between the two domains, but he didn't do it by looking at experiments with weighing mass and seeing if somehow a little bit was not accounted for, and might have slipped over to become energy. Instead he took what seems to be an immensely roundabout path. He seemed to abandon mass and energy entirely, and began to focus on what appeared to be an unrelated topic; \emph{He began to look at the speed of light.}