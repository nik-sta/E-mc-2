%%
% ^2 (squared)
%
% @author: Marcel Stocker
%%

\section{$^2$}

\subsection*{Who was Voltaire?}
\emph{Mr. Voltaire} was in fact \emph{Francois-Marie Arouet} who supported Newton's laws by own means. He thought if Newton can find a rational explanation for the universe, Voltaire could find a rational explanation for the things going on on earth. e.g. the demanded obedience of the king, Aristocrats who got authority from the king which nobody is allowed to question this, the role of money, or other hidden forces in politics. After Francois returned from Paris, three years after that he pushed those ideas. Due to his writings the name Voltaire got more known than \emph{Arouet}. The next thing he wanted to do was to find a place where he can put the seeds of his studies, as it would not be possible to change the nation at once. Which was Emilie du Chatelet.

\subsection*{What was Emilie du Châtelet like?}
Emilie was boy-like (a tomboy). She was strong, fast and had a good intellect which most of the time kept her away from getting friends with others. She was never really into men. More than that, she frightened them away and challenged them etc. she climbed trees and did stuff boys would do. Until she was 19. She then took a husband who was a soldier and was most of the time away. She did that intentionally. And both agreed that she could have affairs while he was out. One of those affairs was a twenty-something year old officer \emph{Pierre-Louis Maupertuis}, who fulfilled her life until he left for a polar expedition. Emilie needed a replacement for him. Then Voltaire came to replace him. They shared the interest in political forms. Emilies husband (Du Chatelet) had a  Chateau (old and abandoned) in Cirey (northern France), which was in his family since Columbus came to America. Emilie and Voltaire then used it for Scientific researches. The funny thing was, when Voltaire ordered the constructors to build a library she ordered them to build a salon, where he planned to place elms, she planted lime trees etc. Finally he still got a Library with seminar areas. The library was  comparable to the Academy of Sciences in Paris. Voltaire did alot of gossip during his researches and was distracted many times by other stuff. Emilie was more eager than Voltaire and was ALMOST about to jump-start future discoveries. Finally she came up with the elementar key question: what is energy?

\subsection*{What was the dispute between Leibniz and Newton about?}
It was about the theory of the movements of objects and what happens when they crash. Newton said that an objects mass times velocity, or their mv1. If a 5-pound ball is going 10 mph, it has 50 units of energy. Leibniz countered: the important factor to focus on was mv2. If a 5-pound ball is going at 10 mph, it has 5 times 100, or 500 units of energy. Beyond this points of views the dispute was also a question of religious beliefs. 

\subsection*{Why was this also a religious conflict?}
Newton believed that god was responsible for the movements of the objects and was responsible for the fact that if two identical objects with identical weight and velocity hit each other frontally, the movement of one object gets stopped to zero by the equal movement of the other object. So the energy of the movement is lost. As if god erased it. Leibniz on the other hand thought that the when the objects hit the energy they had added up. All the energy they carried remained busily in existence, throwing the parts of the objects around. In Leibniz's view nothing is lost. The world runs itself and there is no god controlling this energy flow.

\subsection*{How did du Châtelet prove that Leibniz was right?}
As Leibniz did not provide good evidence for his theory she added up the experiments of a guy named sGravesande, who was not able to publish his research because he was not a theoretician enough. The research of sGravesande was that he let weights fall onto a soft clay floor. The research showed proof for Leibniz equation $E=mv^2$: If Newton's simple $E=mv^1$ was true, then a weight going twice as fast as an earlier one would sink in twice as deeply. One going three times as fast would sink three times as deep. But that's not what 'sGravesande found. If a small brass sphere was sent down twice as fast as before, it pushed four times as far into the clay. If it was flung down three times as fast, it sank nine times as far into the clay. Which is just what thinking of $E=mv^2$ would predict. Soon after her publishing she had a baby and died within a week after due to an infection. This was usual during that time, as the doctors did not know that they need to wash their instruments and had no antibiotics to control infections etc.

\subsection*{Why is squaring the velocity of what you measure such an accurate way to describe what happens in nature?}


\subsection*{What does it mean for mass when c2 is such a large figure?}

\subsection*{8. Mass is simply the ultimate type of condensed or concentrated}
