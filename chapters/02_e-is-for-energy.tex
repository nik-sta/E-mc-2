%%
% e-is-for-energy.tex
%
% @author: Nikola Stankovic
%%
\newpage
\section{E is for Energy}

The word \emph{Energy} is surprisingly new, and can only be traced in its modern sense to the mid 1800s. People before were thought that all the different powers around were unrelated things.

\subsection*{What kind of apprenticeship did Michael Faraday make?}
Michael Faraday was a apprentice bookbinder in the 1810s. And he had no interest in spending his life binding books. His job saved him from the poverty in London. And his job had a singular advantage: ''There were plenty of books there''. Faraday used to read a lot.

\subsection*{How did he get in touch with Humphrey Davy?}
When Faraday was twenty, a shop visitor offered him tickets to a series of lectures at the Royal Institution. Sir Humphrey Davy was speaking on electricity, and on the hidden powers that must exist behind the surface of our visible universe. Faraday made a impressive-looking book with notes and sketches of Davy's lectures and demonstration apparatus, which he sent to him. Davy relied that he wanted to meet Faraday and gave Faraday a job as a lab assistant. But Faradays new position was not as ideal as he'd hoped.

\subsection*{What was the relationship between Humphrey Davy and Michael Faraday like?}
Sometimes Davy behaved as a warm mentor, but at other times he would seem angry, and push Faraday away. It was especially frustrating to Faraday. When Faraday, in his late twenties, was asked to work on how the link between electricity and magnetism might occur he immediately become more cheerful.

\subsection*{How much formal education did Faraday have?}
Faraday wasn't to Oxford, or to Cambridge or, indeed, even attended much of what we call secondary school. He was a relatively uneducated young binder, and also poor, like his father and friends. Faraday's limited formal education, curiously enough, turned out to be a great advantage. This doesn't happen often, because when a scientific subject reaches an advanced level, a lack of education usually makes it impossible for outsiders to get started.

\subsection*{What explains Faraday’s ability to understand the connection between magnetism and electricity?}
Most science students had been trained to show that any complicated motion could be broken down into a mix of pushes and pulls that worked in straight lines. But this approach didn't show how the power of electricity might tunnel trough space to affect magnetism. Because Faraday did not have that bias of thinking in straight lines, he could turn to the Bible for inspiration. The Sandemanian religious group, in which he become an official member, like his family, he belonged to believed in a different geometric pattern: the circle. Farady propped up an magnet. From his religious background, he imagined a whirling tornado of invisible circular lines swirling around it.

\subsection*{What is this? And when was it built?}
\includegraphics[width=0.5\textwidth]{images/image-part-1-question-6.jpg}\par\vspace{1cm}

This is Michael Faraday electric motor and it was built in 1821. Farada's invention was the basis of the electric engine. The full concept of ''Energy'' had still not been formed, but Faraday's discovery that these different kinds of energy were linked was bringing it closer. He made this discovery with just 29 years and it was the high point of his life. Unfortunately Sir Humphry Davy accused Farady of stealing the whole idea and offend him. That Faraday stole the idea was false. Faraday never spoke out against Davy. But for years after the charges of plagiarism and their repercussions, he stayed warily away from front-line-research. Only when Davy died, in 1829, did he get back to work.

\emph{This informations aren't in the book. Informations summarized from the internet.}

\subsection*{What does the Law of the Conservation of Energy mean?}
Faraday's experiments yield that the balancing occurs everywhere. Energy has clearly changed its forms; the system looks very different. Bit the total is exactly, precisely the same. Everything was connected; Everything neatly balanced.

\subsection*{What new energy source did Einstein discover and what effect does this have on the Law of the Conversion of Energy?}

Einstein discovered the new energy source: ?