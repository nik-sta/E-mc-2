%%
% Einstein and the Equation
%
% @author: Marcel Stocker
%%

\section{Einstein and the Equation}

\subsection*{When and where did Einstein publish the equation?}
Einstein published E=mc2 in September 1905. ('Where' is not mentioned)

\subsection*{What material discovered in the 1890s gave hints about the equation?}
metal-streaked ores, which was spraying out some sort of mysterious energy beams.

\subsection*{Who was Marie Curie and how did she die?}
Marie Curie was one of their first investigators, and indeed in 1898 coined the word radioactivity for this active spurting out of radiation. She died of cancer (leukemia), which was produced by the radioactive material (namely: Radium).
In detail: The minute traces of radium powder, which she had carried unknowingly on her blouse and hands as she walked across the muddy cobblestones of 1890s Paris and later, had been pouring out energy in accord with the then-unsuspected equation, barely shrinking at all, for thousands of years. They had been spray-releasing part of themselves without getting used up back when they were deep underground in the Belgian mines in the Congo; they continued through her years of experiments, ultimately giving her this killing cancer. More than seventy years later, the dust would still be alive and could squirt out poisonous radiation onto any archivists who were examining her office ledger, or even the cookbooks at her home.

\subsection*{Why are atomic bombs so powerful?}
Take the great speed of light and square that to get an even more immense number. Then, multiply that by the amount of mass you're looking at, and that's how much energy, exactly, the mass will be able to pour out. But it is easier to access the power of uranium than other material. A uranium bomb works when less than 1 percent of the mass inside it gets turned into energy

\subsection*{What was so ground-breaking and amazing about Einstein’s discovery? (p. 80, 84)}
Mass and energy were never combined. Einstein was like Newton able to produce a complete theory of the physical world.

\subsection*{How precisely did he discover it? (p. 80 top)}
He did not have any labs or tests he could do. He just came to his discovery by dreamingly thinking about it, and by thinking about the discoveries of long dead scientists.

\subsection*{How could you explain the theory of relativity easily? (p. 83)}

\subsection*{What does the term ‘relativity’ NOT mean? (p. 84)}

\subsection*{How did Einstein’s upbringing and background help him discover ‘relativity’?}

\subsection*{How did Einstein’s family life develop as his theory became gradually accepted?}