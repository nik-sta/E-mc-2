%%
% 16_a-brahmin-lifts-his-eyes-unto-the-sky.tex
%
% @author: Jonas Matter
%%

\section{A Brahmin Lifts His Eyes Unto the Sky}

Even though the sun is vast, it can't keep on burning forever. The sun's mass is now 2 * 10$^{27}$ tons, but it consumes about 700 billion tons of its own bulk as hydrogen fuel to keep the multimegaton blasts going each day.

\subsection*{What will happen to the sun in five billion years?}
In a further 5 billion years , the most easily available portions of that fuel will be gone. All that remains at the centre is helium ash. The reactions in our sun will start shifting upward and fuel closer to the surface will be used. The outer layers of the sun will expand and coll down to glow red. The sun will ke on expanding and glowing until it reaches Mercury's orbit. The planet surface will have already melted. The rest will no be absorbed in the flames. A few tens of millions of years later, the same will happen to Venus. What will happen to Earth?

The opinions were divided. Some thought it will end in a great cooling down of the universe. And the other thought that fire and outpourings will take over at the end.

What will happen to Earth is actually both. Any beings left alive in 5 billion years will see the sun get larger and larger until it fills about half the daytime sky. The oceans will boil away and the surface will melt. Life would have to migrate to other planets or survive in deep tunnels but the Earth will have long been barren.

The sun will hold at that great size for another billion years, as the helium ash left inside takes over the burning. Then it will shrink and get weaker and weaker. The burning will not longer be steady. This is what will bring the ice. The sun's surface will sink inward, cause of that the energy output will roar higher again, and the surface will bounce upward again. At this stage, six billion years into our future, it's the final boom of the Titans.

Enough mass is blown away at each bounce upward, that in just a few hundred thousand years, there will be much less of our sun. The sun will lose its gravitational pull. The sun's grip will let the planets go and Earth flies away.

\subsection*{Short answer: What will happen to the sun in five billion years?}
Hydrogen will burn  out in 5 billion years, helium sun expands, too hot on earth, Helium will burn out, Earth will cool, and the sun will lose its Gravitational pull, earth will fly away.

\subsection*{What concept did Chandra (Chandrasekhar) come up with on a trip from India to England in 1930?}
It was known that giant stars can explode, with their top portions rebounding away after they've collided with the core within. But what will happen to the remnant core after that?
Chandra knew that the dense core of a star is under a lot of pressure, and now he began to think about the fact that pressure is a form of energy and energy is another sort of mass.
A compressed star core is under a lot of new pressure, and that pressure can be considered a sort of energy, and wherever there's a concentration of energy, the surrounding space and time will act just as if there's a concentration of mass. So gravity in the remnant star will get more intense due to all this mass (through the pressure). The stronger gravity continues squashing what's left and the pressure can be treated as more energy again. The gravity ratchets up again. It'll result a buildup of pressure. Chandra could now see the insight of E=mc$^2$
Regardless of how hard the substance is at the core, the inside of the star will be crushed. But what will happen to the remaining substance of the star, as it poured into the hole created by this never-ending collapse? Could Chandra be predicting that the inside of the star would disappear. If he was right, then rips were opening up in the very substance of the universe.

By the 1960s there was the first evidence of a start that spins around an area that seems to be entirely empty space. The only thing that would be powerful to do this, is a black hole. In the centre of our galaxy, there's evidence for another black hole, swallowing on average, an ordinary star each year.

Chandra was not able to get backup in public for his work and left England for America and received the Nobel Prize in 1983.

\subsection*{Short answer: What concept did Chandra (Chandrasekhar) come up with on a trip from India to England in 1930?}
Black hole.

\subsection*{What will happen to planet Earth in six billion years?}
Six billion years from now, if Earth is flung loose from the sun, the flight won't be stable through this darker expanse. Our Milky Way is already on track to collide with the Andromeda galaxy and in several billion year, that collision will happen. The turbulence will be enough to shift an Earth's trajectory once more and if Earth slingshots inward, then in a few tens of millions of years we will be within range by the giant black hole at the galaxy's centre. The slingshot outward, will simply delay the end.

By 10$^{18}$ years from now, all galaxies will have emptied out because of such collision. The black holes will travel on their own, sucking mass and energy.

\subsection*{Short answer: What will happen to planet Earth in six billion years?}
Earth will fly away from sun and be swallowed up by a black hole.
