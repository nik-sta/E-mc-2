%%
% Into the Atom
%
% @author: Marcel Stocker
%%

\section{Into the Atom}

\subsection*{How is Ernest Rutherford’s character described?}
He had a common man's accent.
As a student he did not show proper deference to his superiors. But he was the kindest leader of men.

\subsection*{What break-through discovery about the atom did he make?}
That Atoms are almost entirely made of free space. There is just an extremely small concentrated center and a few electrons which flew around them, but in between there is just a lot of free space.

\subsection*{Why did scientists assume that a lot of energy was hidden in the nucleus?}
because it must have been a reason for why the extremely fast electrons were not by its speed to squirmingly escape from the center, but keep the extremely fast electrons in a near distance to the center. Additionally to that, or because of that, there was a big amount of electricity inside the atom (keeping the items together).

\subsection*{Who was James Chadwick and what did he discover?}
James Chadwick was an assistant of James Rutherford. he detected another item in the nucleus (center of the atom): the neutron. It was the same size of the proton (which is in the center) but it was not loaded with electricity, it is neutral. That is why it is called neutron.

\subsection*{Who was Enrico Fermi?}
James Chadwick wanted to try to shoot neutrons into a nucleus so that they would stick there, but with all his test/experiments shooting the neutrons in more and more faster speed he failed. Enrico Fermi an Italian scientist from Rome was the one succeeding. 

\subsection*{What important technique did he provide in 1934?}
He got to the conclusion that if the neutron is too fast it will just slip through the atom/center. He found out, that the shooting neutron must be slowed down in order to stick to the nucleus.
