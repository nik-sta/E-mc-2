%%
% quiet-in-the-midday-snow.tex
%
% @author: Robin Suter
%%

\section{Quiet in the Midday Snow}

\subsection*{Who were Lise Meitner and Otto Hahn?}
\emph{Lise Meitner} was an Austrian who moved to Germany to become one of its leading scientists. Einstein called her "our Madama Curie".
Meitner moved to Berlin in \emph{1907} and quickly became friends with \emph{Otto Hahn}.
Throughout their working relationship, Meitner and Hahn were very formal to each other. Although they never officially dated, Meitner nerver dated anyone else in these years.

In \emph{1934}, when they worked together again (after they switched to different labs), Meitner was fired from the University of Berlin  because she was Jewish. She moved back to Austria.

\subsection*{What happened to Hahn and Meitner’s work in 1938?}
When in 1938 Germany took over Austria, Meitner became a German citizen by default. Kurt Hess, another researcher at the same institute, whispered around that she was Jewish. Shortly after, Hahn asked them to get rid of Meitner. That wasn't nice of him.

Meitner moved to Stockholm. She still remained involved from a distance with the work she had been leading. They tried to stuck neutrons to the nuclei of uranium atoms, but couldn't yet figure out what new substance they were creating with this process.

\subsection*{When did Meitner first meet Einstein and what did she learn there?}
Meitner had first met Einstein at a conference in Salzburg in \emph{1909} where Einstein explained his findings from 1905 that energy could appear out of diappearing mass.

\subsection*{How did Meitner manage to explain the Meitner-Hahn-Strassmann experiments in 1938?}
\emph{Niels Bohr}, who created the most recent model of the nucleus, viewed an atom like a liquid drop. The protons in the nucleus push against each other (both are positively charged). But combined with the electrons, the atom stays stable. Elements with a big nucleus like uranium, the force of the protons is so strong that it can be broken apart by the neutrons that hit the nucleus. Togehter with her nephew, Meitner explained it like a full water balloon. When you squeeze it in the middle, eventually the water will burst out.

She concluded that with the Meitner-Hahn-Strassmann experiments, they cracked the uranium in half. With calculating the energy, they confirmed that the extra energy from the mass was exactly what \emph{\(E=mc^2\)} predicted.

This splitting apart of atoms was labeled \emph{nuclear fission}. Hahn published the findings in Berlin, with minimal credit to Meitner.
