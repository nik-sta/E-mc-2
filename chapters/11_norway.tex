%%
% norway.tex
%
% @author: Robin Suter
%%

\section{Norway}

\subsection*{How did the Allied forces try to sabotage the German bomb project?}
Producing the heavy water required for the nuclear reaction was hard. Instead of building a whole factory, Germany decided to use the power-generating \emph{waterfalls of Norway}.

Norway was resistant to cooperate with the Nazis, but after they destroyed the Norwegian army and asked nicely backed with some machine-guns, the Norwegians agreed.

\subsection*{What happened in the first attack on the Norwegian plant?}
Britain decided this remote factory in Norway was Germany's \emph{weak point} in the construction of the atomic bomb. They attacked it with Airforce bombers, but they \emph{failed} before they even reached the factory (mostly because of bad weather).

\subsection*{What was different in the second sabotage attempt?}
The brits decided on a second attack, this time with soldiers from Norway.
A narrow bridge was thought to be the only way in, but through aerial reconnaissance they found another way in. The workes inside didn't stop them from setting charges. They set the explosives under every one of the eighteen "cells" that separated the heavy water.

The explosion caused the heavy water to flow out and damaged pipes with its shrapnel. The soldiers got away uncaught.
