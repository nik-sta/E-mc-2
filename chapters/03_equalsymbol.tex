%%
% equalsymbol.tex
%
% @author: Nikola Stankovic
%%

\section{=}

\subsection*{When were most typographical symbols created?}
The most of the main typographical symbols were created by the end of the \emph{Middle Ages}.

At the beginning text had looked like telegraphs. Everything was written in uppercase. The first change was to write the most in lowercase. Another shift was to use the point and comma for major and minor breathing pauses.

There was a huge amount of symbols used in the world by different cultures, f. e. all of us know the old Greek alphabet, but also the Hebrew language or the old Egyptian symbols were used in the past. Important is, that it isn't natural that we don't use this symbols instead of ours today. 

Trough the mid-1500s there was still place for entrepreneurs to set their own mark by establishing the remaining minor symbols. \emph{Robert Recorde} found in 1543 the sign "+" later on he invented the symbol for \emph{is equal to} ==========.

There was also other persons, who promoted there signs for \emph{is equal to}. Now a short list, as it might have looked
\begin{itemize}
\item E || mc$^2$
\item E $\longrightarrow$ mc$^2$
\item E .\ae qus. mc$^2$
\item E \big]\big[ mc$^2$
\item E ========== mc$^2$
\end{itemize}


\subsection*{How did Einstein use the = symbol ‘like a telescope’?}
Scientists started using the = symbol as something of a telescope for new ideas. A device for for directing attention to fresh, unsuspected realms. And in this way \emph{Einstein} have used the equals symbol as a telescope.