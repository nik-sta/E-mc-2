%%
% 14_the-fires-of-the-sun.tex
%
% @author: Jonas Matter
%%

\section{The Fires of the Sun}

The flash of light from the explosion over Hiroshima in 1945 reached the orbit of the moon. Some of it bounced back to earth, the rest continued onward to the sun, and then indefinitely beyond.

Our sun explodes the equivalent of many million such bombs every second. So Hiroshima was just a insignificant flicker for the galaxy. E=mc$^2$ does not just apply on earth. Einstein and other physicists had long recognized this. It was just a quirk that the pressures of wartime hat led to the first application in weaponry.

Since the discovery of radioactivity in the 1890s researchers had suspected that uranium or a similar fuel might be operation in the universe, in particular in the sun to keep it burning. But point a spectroscope at the sun: there is no uranium or other radioactively glowing element up there.

What did seem to leap out, was that there was always iron inside them. So in 1909, the best evidence was that the sun was about 66\% pure iron. This was a discouraging result, because Uranium could pour out energy in accord with E=mc$^2$ and iron is different. The nucleus of iron is one of the most perfect and stable. A sphere made of iron could not pour out heat for thousands of millions of year like the sun. 

Suddenly the vision of using E=mc$^2$ to explain the universe was blocked.

Cecilia Payne broke that barrier. She had to face a lot of resistance at Cambride because of her being a woman (the only one in the course). Rutherford was mocking her during the lectures. A woman couldn't do graduate work in this field in England and so she went to Harvard. She was bursting with enthusiasm. But that's dangerous because that usually means you're trying to fit in with their ideas and approaches. Researches should keep a critical distance. Payne hat that distance. But there was still a ordeal of sexism at Haward. 
Payne wrote a Ph.D. that would let her confirm and further develop an new theory about how to build up spectroscope interpretations.
She checked for ambiguities in spectroscope lines.
Payne's interpretation was that there is over 90\% hydrogen in the sun, and most of the rest being the nearly as lightweight helium.
That would change what was understood about how stars burn. Iron is so stable that no one could imagine it transformed through E=mc$^2$ to generate heat in the sun. But what wold hydrogen do? The old guard knew. Their career depended on the fact, that there should be iron in the sun. Her thesis adviser declared her wrong. Even her old thesis adviser (Russel) declared her wrong. If she wanted to get her reaseach accepted she'd have to recant and say that the enormous abundance of hydrogen is certainly not real. A few years later when independent researchers backed her spectroscope reinterpretations, Payne was vindicated.
The way was now open to applying E=mc$^2$ to explain the fires of the sun. She had shown that the sun and the stars are great E=mc$^2$ pumping stations. The seem to squeeze hydrogen mass out of existence. But in fact they're simply squeezing it along E=mc$^2$, so that what hat appeared as mass now bursts into the form of explosive energy.
Down on earth, hydrogen atoms won't stick to each other. But trapped near the center of the sun, under thousands of miles of weighty substance overhead, hydrogen nuclei can join together to become the element helium.
The mass of 4 hydrogen nuclei can be written as 4. When they join together as
helium, the weight is 0.7 percent less. That missing weight comes out as roaring energy. The reason the sun is so much more powerful (than the uranium bomb over Japan) is that it pumps 4 million tons of hydrogen into pure energy each second.


\subsection*{What is the topic of this chapter? (p. 173)}

How the equation controls how stars ‘work’ and how life will end.
It shows how the equation's sway extends throughout the universe: controlling everything from how the first stars ignited, to how life will end.

\subsection*{What did astronomers believe about the sun's content at the time Einstein discovered that E=mc$^2$?}

The sun is made up of 66\% iron.

\subsection*{How does a spectroscope work?}

Every element gives off a distinct visual signal: spectroscope breaks the spectrum of electromagnetic radiation into various wavelengths allowing them to be identified.

\subsection*{What was Cecilia Payne’s academic career?}

University of Cambridge (England), switched majors multiple times, a woman couldn't do graduate work in this field in England and went to Harvard, wrote a Ph.D..

\subsection*{How did Payne come up with a different interpretation of the spectroscope lines?}

She reinterpreted the spectroscope lines (different amount of ionization at different temperatures): the sun consists mostly (90\%) of Hydrogen.

\subsection*{How did the astronomy establishment react to her findings?}

The old guards career depended on the fact, that there should be iron in the sun. Her thesis adviser declared her wrong. A few years later when independent researchers backed her spectroscope reinterpretations, Payne was vindicated.

\subsection*{What happens inside the sun to release so much energy?}

Hydrogen atoms fuse to one helium atom, releasing lots of energy in the process.