%%
% 13_0816-am-over-japan.tex
%
% @author: Jonas Matter
%%

\section{8:16 A.M. - Over Japan}

The bomb (made in New Mexico), an elongated (stretched) trash can, had taken 43 seconds to fall from B-29 and hit the ground.

There were small holes in the middle where wires had been tugged out of it as it dropped away to start the clock switches of its first arming system.

More small holes were farther back to took in samples of air during the fall. As soon as the bomb was under 7000 feet above the ground a barometric switch was turned, priming the second arming system.

The bomb was 3m long and 80cm wide, so impossible to see from the ground.

Weak radio signals were being pumped down from the bomb to the ground. Some signals were absorbed by the hospital walls directly below but most of it bounced back to the bomb. The time lag was used to measure the height remaining to the ground. The last signal arrived at 1900ft. 

John von Neumann had calculated that a bomb exploding much higher than 2000ft. would dissipate much of its heat in open air. Exploding much lower would dig a huge crater. At just under 2000ft. the height would be ideal.

\subsection*{What happened inside the bomb when it was triggered?}

An electric impulse lit cordite sacs, producing a conventional artillery blast. A small part of the purified uranium was now pushed down a gun barrel inside the bomb.
The first uranium segment impacted the remaining bulk of the uranium. There were a number of stray and loose neutrons inside it. Although the uranium atoms were protected by electrons, the escaped neutrons (having no electrical charge), were not affected by the electrons. While many of them flew straight through out the other side, a few where on a collision course for the speck of a nucleus far down the centre. 

That nucleus blocked outside particles normally, for it was seething with positively charged protons. But since neutrons have no electric charge they're invisible to the protons. 

The arriving neutrons pushed into the nucleus, overbalancing it, making it wobble. Once the wobbling in the nucleus was enough to break the strong force glue, then the ordinary electricity of the protons was available to force them apart. Its speeding impact into the other parts of the uranium didn't heat it up much. But the density of uranium was enough that a chain reaction started.

So there wasn't just two fragments of uranium nuclei, there were four, then eight, and so on. Mass was "disappearing" within the atoms and coming out as the energy of speeding nuclei fragments. E=mc$^2$ was now under way. The chain reaction of multiplying releases was finished in barely a few millionths of a second. It went through 80 generations of doubling before it ended. By the last few, the segments of broken uranium nuclei were moving so fast, that the started heating up the metal around them. From this point on, all the action of the E=mc$^2$ reaction was over. No more energy appeared. The energy in the movement of those nuclei was simply being transformed to heat energy (just as rubbing hands together). The uranium fragments were rubbing against resting metal at immense speed. The rubbing made the metals inside the bomb begin to warm. Because the generations of chain reaction doubling had gone on, it got warmer and warmer to several million degrees (temperature of the centre of the sun). The heat move out. It goes through the steel tamping around the uranium and the massive casing of the bomb, but then it pauses.

\subsection*{What happens on the ground and what is the condition inside the triggered bomb compared to?}
Entities as hot as that explosion have energy that must be released. It starts pushing X rays out of itself. The explosion is hovering, the fragments are trying to cool themselves off. They remain that way, pouring out large part of their energy. Then when the X ray spraying is over, the heat ball resumes its outward spread. Only now does the central eruption become visible. An object resembling a sun now appears. The object burns at full power for about 0.5 seconds, then begins to fade away, taking 2-3 seconds to empty itself out. The emptying is accomplished by spraying heat energy outward. Fires begin, skin explodes off. The first of the tens of thousands of deaths in Hiroshima begin. 

A third of the energy from the chain reactions comes out in this flash. The rest follows behind. The objects heat pushes on ordinary air, accelerating it to speeds that have never occurred. 

After that heat wave there's a second air pulse, a little slower. After the atmosphere sloshes back. This lowers the air density to virtually zero. Life-forms that survived the heat wave will now explode outward.

A small amount of the heat that was produced cant move forward. In a few seconds it begins to rise. It swells as it goes, and at a sufficient height it spreads out producing the mushroom.

\subsection*{Short answer: What happened inside the bomb when it was triggered?}
Short answer: A chain reaction

\subsection*{Why was it triggered at 2000 ft. above ground?}
John von Neumann had calculated that a bomb exploding much higher than 2000ft. would dissipate much of its heat in open air. Exploding much lower would dig a huge crater. At just under 2000ft. the height would be ideal.

\subsection*{Short answer: What is the condition inside the triggered bomb compared to?}
The centre of the sun.

\subsection*{How much heat is produced?}
The rubbing and battering made the metals inside the bomb begin to warm to boiling temperature. But the generations of chain reaction doubling had gone on, as more uranium atoms had been splitting, it will reach several million degrees and then it kept on rising.


\subsection*{Short answer: What happens on the ground?}
Heat wave, vacuum and an incredibly strong storm, radioactive fall-out