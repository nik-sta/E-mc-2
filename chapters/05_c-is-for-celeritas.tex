%%
% 05_c-is-for-celeritas.tex
%
% @author: Patrick Scherler
%%
\newpage
\section{c is for celeritas}

\subsection*{1. Who was the first scientist to try to measure the speed of light?}
Almost everyone was convinced that light travelled \emph{infinitely fast}. \emph{Galileo} was the first person to clearly conceive of measuring the speed of light, but he was too old to carry out the experiment himself. The experiment was a good idea, but the technology of the time was too poor and the distance too short to get any clear result.

\subsection*{2. How did Ole Roemer discover the speed of light?}
He analysed the movement of the moon Io of Jupiter: It was supposed to travel around its planet every 42.5 hours. But it never stuck honestly to schedule. Roemer assumed that the problem was in how Earth traveled. In the summer, for example, if Earth was closer to Jupiter, the light's journey would be shorter, and the moon's image would arrive sooner. Because of the great distance he was able to get precise results.

\subsection*{3. What is the explanation given for the fact that Roemer discovered the speed of light?}

\subsection*{4. How did Cassini react to Roemer's success?}
Cassini declared he had not been proven wrong at the challenge and he had lots of supporters. Roemer had performed an impeccable experiment, with a clear prediction, yet Europe's astronomers still did not accept that light traveled at a finite speed.

\subsection*{5. How fast is the speed of light?}
The value he had estimated for light's speed was close to the best current estimate, which is about 670'000'000 mph (300'000 km/s)

\subsection*{6. In what ways were Faraday and Maxwell similar?}
They were both deeply religious men. Maxwell was bullied in school when he was younger. He never expressed any anger about it. Faraday also still carried the wounds from his experiences with Sir Humphry Davy in the 1820s.

Maxwell was such a great mathematician that he was able to see beyond the surface simplicity of Faraday's sketches.

\subsection*{7. What is the inner property of light?}
A light beam consists of electromagnetic waves.

What was happening inside a light beam, Maxwell began to see, was just another variation of the back-and-forth movement. When a light beam starts going forward, one can think of a little bit of electricity being produced, and then as the electricity moves forward it powers up a little bit of magnetism, and as that magnetism moves on, it powers up yet another surge of electricity, and so on.

\subsection*{8. In what ways is light different from other movement?}
A surfer's water wave can appear to hold still, because all the parts of the wave take up a steady position in relation to one another.

Einstein concluded that light can exist only when a light wave is actively moving forward.

\subsection*{9. Why is it impossible to catch up with the speed of light?}
Whenever you think you're racing forward fast enough to have pulled up next to a light beam, look harder and you'll see that whatever part you thought you were close to is powering up a further part of the light beam that is still hurtling away from you.

\subsection*{10. Why is -273� the coldest possible temperature?}

\subsection*{11. What happens to matter approaching the speed of light?}
The energy of speed will become mass, so the matter will expand.



\subsection*{13. Why did nobody before Einstein notice the connection between mass and energy?}
The speed of light is so much higher than the ordinary motions we're used to.

The effect is weak at walking speed, or even at the speed of locomotives or jets, but it's still there.